% Use only LaTeX2e, calling the article.cls class and 12-point type.

\documentclass[12pt]{report}
\usepackage[a4paper]{geometry}
\usepackage[myheadings]{fullpage}
\usepackage{fancyhdr}
\usepackage{lastpage}
\usepackage{graphicx, wrapfig, subcaption, setspace, booktabs}
\usepackage[T1]{fontenc}
\usepackage[font=small, labelfont=bf]{caption}
\usepackage{fourier}
\usepackage[protrusion=true, expansion=true]{microtype}
\usepackage[english]{babel}
\usepackage{sectsty}
\usepackage{url, lipsum}
\usepackage{tgbonum}
\usepackage{hyperref}
\usepackage{xcolor}
\usepackage{dingbat}
\usepackage{comment}
\pagestyle{fancy}
\setlength\headheight{15pt}
\fancyhead[L]{}
\fancyhead[R]{\url{https://github.com/balakrishnankom/SOEN6481}}


% Users of the {thebibliography} environment or BibTeX should use the
% scicite.sty package, downloadable from *Science* at
% www.sciencemag.org/about/authors/prep/TeX_help/ .
% This package should properly format in-text
% reference calls and reference-list numbers.



% Use times if you have the font installed; otherwise, comment out the
% following line.

\usepackage{times}

% The preamble here sets up a lot of new/revised commands and
% environments.  It's annoying, but please do *not* try to strip these
% out into a separate .sty file (which could lead to the loss of some
% information when we convert the file to other formats).  Instead, keep
% them in the preamble of your main LaTeX source file.


% The following parameters seem to provide a reasonable page setup.

\topmargin 0.0cm
\oddsidemargin 0.1cm
\textwidth 16cm 
\textheight 22cm
\footskip 1.0cm


%The next command sets up an environment for the abstract to your paper.

\newenvironment{sciabstract}{%
\begin{quote} \bf}
{\end{quote}}




% If your reference list includes text notes as well as references,
% include the following line; otherwise, comment it out.



% The following lines set up an environment for the last note in the
% reference list, which commonly includes acknowledgments of funding,
% help, etc.  It's intended for users of BibTeX or the {thebibliography}
% environment.  Users who are hand-coding their references at the end
% using a list environment such as {enumerate} can simply add another
% item at the end, and it will be numbered automatically.

\newcounter{lastnote}
\newenvironment{scilastnote}{%
\setcounter{lastnote}{\value{enumiv}}%
\addtocounter{lastnote}{+1}%
\begin{list}%
{\arabic{lastnote}.}
{\setlength{\leftmargin}{.22in}}
{\setlength{\labelsep}{.5em}}}
{\end{list}}



 


% Include your paper's title here
\title{Universal Parabolic Constant}

\author{Balakrishnan Rajagopal (40075977) }
\date{}
\renewcommand{\baselinestretch}{1.15}
% Include the date 
\date{}

\begin{document} 

\maketitle 

% Double-space the manuscript.

\baselineskip20pt

% Make the title.

% Place your abstract within the special {sciabstract} environment.
\tableofcontents
\chapter{Repository Address}
\url{https://github.com/balakrishnankom/SOEN6481}


\chapter{User Stories}
    
 The Fibonacci Sequence was used for user story estimation.

\begin{quote}
                \section{US-N10-1 - Calculate Universal Parabolic Constant}
                
                \begin{tabular}{ |p{4cm}|p{10cm}| }
                 \hline
                 \multicolumn{2}{|c|}{\textbf{US-N10-1 - Calculate Universal Parabolic Constant}} \\
                 \hline
                 \textbf {Story ID}& US-N10-1  \\
                 \hline
                 \textbf{Priority} & Must have \\
                 \hline
                 \textbf{Description}   & As a user, I should be able to calculate the universal parabolic constant by clicking a button so that I can use it for other calculations.
  \\
                 \hline
                 \textbf{Acceptance}& 
                
                 I know I'm done when I click the UPC button, constant is calculated.
                
                \\
                 \hline
                 \textbf{Estimate} &  5  point  \\
                 \hline
                 \textbf{Constrains}& Universal Parabolic constant should be displayed or should be appended to the computation.  \\
                \hline
                
                \hline
                \end{tabular}
            \hfill\break\\\\
                
        
        
        
        
        \section{US-N10-2 - Select the number of precisions of the constant}
                \begin{tabular}{ |p{4cm}|p{10cm}| }
                 \hline
                 \multicolumn{2}{|c|}{\textbf{US-N10-2 - Select the number of precisions of the constant}} \\
                 \hline
                 \textbf {Story ID}& US-N10-2  \\
                 \hline
                 \textbf{Priority} & Must have \\
                 \hline
                 \textbf{Description}   &  As a user, I should be able to choose the number of precisions of the constant so that I can get the desired result.   \\
                 \hline
                 \textbf{Acceptance}& I know I am done When the result has only the selected number of the digits after decimal point.
                
                \\
                 \hline
                 \textbf{Estimate} &  2 points  \\
                 \hline
                 \textbf{Constrains}& The number of digits should be a positive number.   \\
                 \hline
                     \hline
                \end{tabular}
            \hfill\break\\\\
    

       
              \section{US-N10-3 - Symbolize the constant}
                \begin{tabular}{ |p{4cm}|p{10cm}| }
                 \hline
                 \multicolumn{2}{|c|}{\textbf{US-N10-3 - Symbolize the constant} } \\
                 \hline
                 \textbf {Story ID}& US-N10-3  \\
                 \hline
                 \textbf{Priority} & Won't have \\
                 \hline
                 \textbf{Description}   & As a user, I should be able to choose to display the constant as a symbol so that the computation will be easy to edit. \\
                 \hline
                 \textbf{Acceptance}& 
                
                 I know I am done when, I click the UPC constant button, 'P' symbol is displayed instead of 2.29558714939 \\
                 \hline
                 \textbf{Estimate} &  2 points  \\
                 \hline
                 \textbf{Constrains}& The calculator should display the symbol for further computations.  \\
                 \hline
                 \hline
                \end{tabular}
            \hfill\break\\
            
             \section{US-N10-4 - Store result}
                \begin{tabular}{ |p{4cm}|p{10cm}| }
                 \hline
                 \multicolumn{2}{|c|}{\textbf{US-N10-4 - Store result} } \\
                 \hline
                 \textbf {Story ID}& US-N10-4  \\
                 \hline
                 \textbf{Priority} & Should have \\
                 \hline
                 \textbf{Description}   & As a user,I want the result of the computation to be stored in memory so that It could be used for next calculation.  \\
                 \hline
                 \textbf{Acceptance}& 
                
                 I know I am done when,  When I am able to retrieve the result of the last computation. \\
                 \hline
                 \textbf{Estimate} &  3  points  \\
                 \hline
                 \textbf{Constrains}& only the result of the last computation should be stored.   \\
                 \hline
                            \hline
                \end{tabular}
            \hfill\break\\
            
            
            
               \section{US-N10-5 - Arithmetic Operations}
                \begin{tabular}{ |p{4cm}|p{10cm}| }
                 \hline
                 \multicolumn{2}{|c|}{\textbf{US-N10-5 - Arithmetic Operations} } \\
                 \hline
                 \textbf {Story ID}& US-N10-5  \\
                 \hline
                 \textbf{Priority} & Must have \\
                 \hline
                 \textbf{Description}   & As a user, I should be able to perform all arithmetic functions such as add,subtract,multiply and division, so that I can use the arithmetic operations in other computations. \\
                 \hline
                 \textbf{Acceptance}& 
                
                 I know I am done when,  when the result of the chosen arithmetic operation is displayed. \\
                 \hline
                 \textbf{Estimate} &  3  points  \\
                 \hline
                 \textbf{Constrains}& The result of the arithmetic operation should be displayed in less than one second \\
                 \hline
                 
                \end{tabular}
            \hfill\break\\
            
              \section{US-N10-6 - Editable input}
                \begin{tabular}{ |p{4cm}|p{10cm}| }
                 \hline
                 \multicolumn{2}{|c|}{\textbf{US-N10-6 - Editable input} } \\
                 \hline
                 \textbf {Story ID}& US-N10-6  \\
                 \hline
                 \textbf{Priority} & Should have \\
                 \hline
                 \textbf{Description}   & As a user, I should be able to edit the computation so that I can change the operator and the operands. \\
                 \hline
                 \textbf{Acceptance}& 
                
               I know I am done when, I click the input field, I should be able to modify the operator and operands. \\
                 \hline
                 \textbf{Estimate} &  2 points  \\
                 \hline
                 \textbf{Constrains}&  When the user is able to edit, only the numbers or allowed symbols should be given as input.   \\
                 \hline
                \end{tabular}
            \hfill\break\\
            
            
             
        
              \section{US-N10-7 - Clear the screen}
                \begin{tabular}{ |p{4cm}|p{10cm}| }
                 \hline
                 \multicolumn{2}{|c|}{\textbf{US-N10-7 - Clear the screen} } \\
                 \hline
                 \textbf {Story ID}& US-N10-7  \\
                 \hline
                 \textbf{Priority} & Should have \\
                 \hline
                 \textbf{Description}   &As a user, I should be able to clear the screen so that I can proceed with next computation.\\
                 \hline
                 \textbf{Acceptance}& 
                
               I know I am done when, I click the clear button, input field or the display should be cleared. \\
                 \hline
                 \textbf{Estimate} &  1 points  \\
                 \hline
                 \textbf{Constrains}&  When the user clicks clear button, the display should be cleared.   \\
                 \hline
                \end{tabular}
            \hfill\break\\
            
        
        
        
             
        
              \section{US-N10-8 - Calculate the area }
                \begin{tabular}{ |p{4cm}|p{10cm}| }
                 \hline
                 \multicolumn{2}{|c|}{\textbf{US-N10-8 - Calculate the area} } \\
                 \hline
                 \textbf {Story ID}& US-N10-8  \\
                 \hline
                 \textbf{Priority} & Must have \\
                 \hline
                 \textbf{Description}   &As a user, I should be able to calculate the area of a parabolic arch given the height and chord so that I can use it for my application.\\
                 \hline
                 \textbf{Acceptance}& 
                
               I know I am done when, I click the area button, area of the parabolic arch is displayed. \\
                 \hline
                 \textbf{Estimate} &  5 points  \\
                 \hline
                 \textbf{Constrains}&  Height and Chord should be positive.   \\
                 \hline
                \end{tabular}
            \hfill\break\\
            
        
            
        \end{quote}
     
        
      
        

        
      


    
 \newpage

\chapter{Backward Traceability Matrix}


\begin{quote}

\hfill

\begin{tabular}{|p{2cm}|p{4cm}|p{2cm}|p{2cm}|p{2cm}|p{2cm}|}

\hline 
\textbf{US ID}&\textbf{US Name}& \textbf{Interviewee} & \textbf{Online Sources} & \textbf{Domain Modal}&\textbf{Use Case}\\
\hline

US-N10-1&Calculate Universal Parabolic Constant&\checkmark&\checkmark&\checkmark&\checkmark\\
\hline
US-N10-2&Select the number of precisions of the constant&\checkmark&\checkmark&&\\
\hline
US-N10-3&Symbolize the constant&\checkmark&\checkmark&&\\
\hline

US-N10-4&Store result&\checkmark&&\checkmark&\\
\hline

US-N10-5&Arithmetic Operations&\checkmark&\checkmark&\checkmark&\checkmark\\
\hline

US-N10-6&Editable input&\checkmark&&&\\
\hline

US-N10-7&Clear the screen&\checkmark&&&\\
\hline


US-N10-8&Calculate the area&&\checkmark&&\\
\hline

\hline


\end{tabular}
\end{quote}


\chapter{Implementation}

The Universal Parabolic Constant calculator is implemented using Java without using built-in libraries. The Source of the calculator can be found on the repository (Chapter 1) mentioned above.

\section{User Stories Implemented}

The following user stories were implemented in the UPC calculator, the User stories which have been implemented are chosen based on the priority.

\begin{enumerate}
    \item\textbf{ US-N10-1} - Calculate Universal Parabolic Constant
    \item \textbf{US-N10-2} - Select the number of precision of the constant.
    \item \textbf{US-N10-4} - Store result
    \item \textbf{US-N10-5} - Arithmetic Operations
    \item \textbf{US-N10-7} - Clear the screen
    \item \textbf{US-N10-10} - Calculate the area
\end{enumerate}




\chapter{References}
\begin{quote}
    

\begin{enumerate}
\item Reese, Sylvester and Sondow, Jonathan. "Universal Parabolic Constant." From MathWorld--A Wolfram Web Resource, created by Eric W. Weisstein. http://mathworld.wolfram.com/UniversalParabolicConstant.html
\item https://keisan.casio.com/exec/system/1223291032
\item Sylvester Reese and Jonathan Sondow, Feb 13 2005 "https://oeis.org/A103710"
\item https://www.revolvy.com/page/Universal-parabolic-constant

\end{enumerate}


\end{quote}

\end{document}


















