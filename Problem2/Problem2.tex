\documentclass{report}
\usepackage[utf8]{inputenc}
\usepackage{hyperref}
\usepackage{setspace}
\usepackage{float}
\title{Universal Parabolic Constant}

\author{Balakrishnan Rajagopal (40075977) }
\date{}
\renewcommand{\baselinestretch}{1.15}

\usepackage{natbib}
\usepackage{graphicx}

\usepackage[final]{pdfpages}


\usepackage{fancyhdr}
\usepackage[margin=1in]{geometry}
\usepackage{pdfpages}
\renewcommand{\headrulewidth}{0.1pt}
\fancyhf{} 
\renewcommand{\headrulewidth}{0.1pt} 
\fancyfoot[C]{\thepage} 
\renewcommand{\footrulewidth}{0.1pt} 


\begin{document}
\maketitle
\chapter{Interview}



\section{Interviewee}
\newline Name    : Srividya Murugan
\newline Email : srividya92.sv@gmail.com
\newline Background : She has done Master's in Applied Mathematics.

\newenvironment{qanda}{\setlength{\parindent}{0pt}}{\bigskip}
\newcommand{\Q}{\bigskip\bfseries Q: }
\newcommand{\A}{\par\textbf{A:} \normalfont}
 
\section{Questions \& Answers}
\begin{qanda}
 
\Q How long have you been working with constants?
\A 3 years

\Q Why do we need constants in Math?
\A Mathematical Constants have the value which is fixed by a definition. It will be used across multiple mathematical problem.

\Q How is Universal Parabolic Constant applied?
\A It is not widely used. But it is used in specific complex applications. Eg. Parabolic Bridge


\Q How is Universal Parabolic Constant applied?
\A It is not widely used. But it is used in specific complex applications. Eg. Parabolic Bridge (Physics).

\Q What is the constant's value?
\A It is a fixed value that can be derived from the below equation.

\begin{equation}
P = \ln(1+\sqrt{2})+\sqrt{2} = 2.29558714939
\end{equation}

\Q How often you use this constant in your research area?
\A I use this constant in my study frequently.

\Q Do you use calculator often?
\A Yes.

\Q Do you find it hard to use the irrational numbers while using the calculator? 
\A Yes. It is quite difficult to type irrational numbers with the other numbers. It is not easy to double check.

\Q  Will integrating this constant in the calculator make your calculation easy? 
\A Yes. 

\Q  Do you want to display the symbol or the irrational number in the calculator?
\A  Symbol would be fine.






\section{Analysis}

It is obvious that Mathematicians need an efficient way to use these constants in the calculator. They find it hard to type irrational numbers in the calculator since they use these numbers often in their study. They need these constants in the calculator which would make their calculation easy. 



\end{qanda}






\bibliographystyle{plain}
\bibliography{references}
\end{document}